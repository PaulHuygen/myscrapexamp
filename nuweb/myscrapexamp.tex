\newcommand{\NWtarget}[2]{#2}
\newcommand{\NWlink}[2]{#2}
\newcommand{\NWtxtMacroDefBy}{Fragment defined by}
\newcommand{\NWtxtMacroRefIn}{Fragment referenced in}
\newcommand{\NWtxtMacroNoRef}{Fragment never referenced}
\newcommand{\NWtxtDefBy}{Defined by}
\newcommand{\NWtxtRefIn}{Referenced in}
\newcommand{\NWtxtNoRef}{Not referenced}
\newcommand{\NWtxtFileDefBy}{File defined by}
\newcommand{\NWtxtIdentsUsed}{Uses:}
\newcommand{\NWtxtIdentsNotUsed}{Never used}
\newcommand{\NWtxtIdentsDefed}{Defines:}
\newcommand{\NWsep}{${\diamond}$}
\newcommand{\NWnotglobal}{(not defined globally)}
\newcommand{\NWuseHyperlinks}{}



\documentclass[twoside]{artikel3}
\pagestyle{headings}
\usepackage{pdfswitch}
\usepackage{figlatex}
\usepackage{makeidx}
\renewcommand{\indexname}{General index}
\makeindex
\newcommand{\thedoctitle}{Scraper example}
\newcommand{\theauthor}{Paul Huygen <paul.huygen@huygen.nl>}
\newcommand{\thesubject}{Scraper example}
\title{\thedoctitle}
\author{\theauthor}
\date{\today \\ 09:30~h.}
% Packages.
\usepackage{a4wide}
\usepackage{alltt}
\usepackage{color}
\usepackage{lmodern}
\usepackage[latin1]{inputenc}
\usepackage[T1]{fontenc}
\usepackage[british]{babel}
%
% Commands for frequently used constructions
%
\newcommand{\pdf}{\textsc{pdf}}
\newcommand{\HTML}{\textsc{html}}
\newcommand{\URI}{\textsc{uri}}
\newcommand{\URL}{\textsc{url}}
%
% PDF-specific settings
%
\ifpdf
% \usepackage[pdftex]{graphicx}       %%% graphics for dvips
% \usepackage[pdftex]{thumbpdf}      %%% thumbnails for ps2pdf
% \usepackage[pdftex]{thumbpdf}      %%% thumbnails for pdflatex
% \usepackage[pdftex,                %%% hyper-references for pdflatex
% bookmarks=true,%                   %%% generate bookmarks ...
% bookmarksnumbered=true,%           %%% ... with numbers
% a4paper=true,%                     %%% that is our papersize.
% hypertexnames=false,%              %%% needed for correct links to figures !!!
% breaklinks=true,%                  %%% break links if exceeding a single line
% linkbordercolor={0 0 1}]{hyperref} %%% blue frames around links
% %                                  %%% pdfborder={0 0 1} is the
% %                                  default
% \hypersetup{
%   pdfauthor   = {\theauthor},
%   pdftitle    = {\thedoctitle},
%   pdfsubject  = {web program},
%  }
 \renewcommand{\NWlink}[2]{\hyperlink{#1}{#2}}
 \renewcommand{\NWtarget}[2]{\hypertarget{#1}{#2}}
 \renewcommand{\NWsep}{$\diamond$\rule[-1\baselineskip]{0pt}{1\baselineskip}}
\else
%\usepackage[dvips]{graphicx}        %%% graphics for dvips
%\usepackage[latex2html,             %%% hyper-references for ps2pdf
%bookmarks=true,%                   %%% generate bookmarks ...
%bookmarksnumbered=true,%           %%% ... with numbers
%hypertexnames=false,%              %%% needed for correct links to figures !!!
%breaklinks=true,%                  %%% breaks lines, but links are very small
%linkbordercolor={0 0 1},%          %%% blue frames around links
%pdfborder={0 0 112.0}]{hyperref}%  %%% border-width of frames 
\usepackage{html}
\renewcommand{\NWlink}[2]{\hyperlink{#1}{#2}}
\renewcommand{\NWtarget}[2]{\hypertarget{#1}{#2}}
\fi
%
% Settings
%
\raggedbottom
\makeatletter
\if@oldtoc
  \renewcommand\toc@font[1]{\relax}
\else
  \renewcommand*\toc@font[1]{%
    \ifcase#1\relax
    \chaptocfont
    \or\slshape
    \or\rmfamily
    \fi}
\fi
\makeatother
\newcommand{\chaptocfont}{\large\bfseries}

\newcommand{\pdfpsinc}[2]{%
\ifpdf
  \input{#1}
\else
  \input{#2}
\fi
}
\begin{document}
\maketitle
\begin{abstract}
  In this document a web-scraper is constructed that scrapes the forum
  \url{ragingbull.com}, using Python and Beautifulsoup.
\end{abstract}
\tableofcontents

\section{Introduction}
\label{sec:Introduction}

\begin{itemize}
\item Scrape a forum on a website.
\item In this case \url{ragingbull.com}.
\item Use Python and Beautifulsoup.
\end{itemize}

\subsection{Structure of the forum}
\label{sec:forumstructure}

The forum consists of a set of \emph{boards} with different
subjects. Each board has an identifying number and a name, e.g. board
14242 is about \emph{Current Events}, abbreviated as
\textsc{cevt}. The main page of that board has as \textsc{url}:
\url{ragingbull.com/board/14242}. It contains a table with a list
of topics and, when there are too many topics for a single page, references to other \textsc{url}'s that contain lists of
older topics. These \textsc{url}'s look like
\url{ragingbull.com/board/14242/page/2}.

A topic has as url e.g. \url{ragingbull.com/topic/1061702} and a title. The
page of the topic contains a list of posts. 

\subsection{What are we going to do?}
\label{sec:what}

\begin{enumerate}
\item Read the pages of the board and collect the url's of the topics
\item Read the pages of the topics and extract the posts.
\item Wrap each post (text and metadata) in a \textsc{naf} file.
\end{enumerate}

\subsection{Metadata}
\label{sec:metadata}

We need to collect for each post the following metadata:
\begin{enumerate}
\item board name and ID.
\item Topic name and ID.
\item Sequence number of the post in the topic.
\item Author ID.
\item Date of the post.
\end{enumerate}

To test whether we have gathered a post with the correct metadata, we
can print it as follows:

\begin{flushleft} \small
\begin{minipage}{\linewidth}\label{scrap1}\raggedright\small
\NWtarget{nuweb?}{} $\langle\,${\itshape methods of the main program}\nobreak\ {\footnotesize {?}}$\,\rangle\equiv$
\vspace{-1ex}
\begin{list}{}{} \item
\mbox{}\verb@def print_post(board_id, board_name, topic, seq, author, post_date, text):@\\
\mbox{}\verb@    print( "Board:   {} ({})".format(board_id, board_name))@\\
\mbox{}\verb@    print( "Topic:   {}".format(topic))@\\
\mbox{}\verb@    print( "Post nr: {}".format(seq))@\\
\mbox{}\verb@    print( "Date:    {}".format(post_date))@\\
\mbox{}\verb@    print( "Text: {}".format(text))@\\
\mbox{}\verb@@\\
\mbox{}\verb@@{\NWsep}
\end{list}
\vspace{-1.5ex}
\footnotesize
\begin{list}{}{\setlength{\itemsep}{-\parsep}\setlength{\itemindent}{-\leftmargin}}
\item \NWtxtMacroRefIn\ \NWlink{nuweb?}{?}.
\item \NWtxtIdentsDefed\nobreak\  \verb@print_post@\nobreak\ \NWlink{nuweb?}{?}.\item \NWtxtIdentsUsed\nobreak\  \verb@print@\nobreak\ \NWlink{nuweb?}{?}.
\item{}
\end{list}
\end{minipage}\vspace{4ex}
\end{flushleft}
\section{The program}
\label{sec:program}



\subsection{Read the command-line}
\label{sec:read-commandline}

ˆThe title of this section is mesleading, because in this demo-phase
we are not going to read the command-line. We just read a
pre-programmed forum board.

\begin{flushleft} \small
\begin{minipage}{\linewidth}\label{scrap2}\raggedright\small
\NWtarget{nuweb?}{} $\langle\,${\itshape get program options}\nobreak\ {\footnotesize {?}}$\,\rangle\equiv$
\vspace{-1ex}
\begin{list}{}{} \item
\mbox{}\verb@@\\
\mbox{}\verb@boardnum = 14242@\\
\mbox{}\verb@@\\
\mbox{}\verb@@{\NWsep}
\end{list}
\vspace{-1.5ex}
\footnotesize
\begin{list}{}{\setlength{\itemsep}{-\parsep}\setlength{\itemindent}{-\leftmargin}}
\item \NWtxtMacroRefIn\ \NWlink{nuweb?}{?}.
\item \NWtxtIdentsDefed\nobreak\  \verb@boardnum@\nobreak\ \NWlink{nuweb?}{?}.
\item{}
\end{list}
\end{minipage}\vspace{4ex}
\end{flushleft}
\subsection{The program file}
\label{sec:program-file}

\begin{flushleft} \small
\begin{minipage}{\linewidth}\label{scrap3}\raggedright\small
\NWtarget{nuweb?}{} \verb@"../scrape.py"@\nobreak\ {\footnotesize {?}}$\equiv$
\vspace{-1ex}
\begin{list}{}{} \item
\mbox{}\verb@@\\
\mbox{}\verb@@\hbox{$\langle\,${\itshape import modules in main program}\nobreak\ {\footnotesize \NWlink{nuweb?}{?}}$\,\rangle$}\verb@@\\
\mbox{}\verb@@\hbox{$\langle\,${\itshape methods of the main program}\nobreak\ {\footnotesize \NWlink{nuweb?}{?}}$\,\rangle$}\verb@@\\
\mbox{}\verb@@\\
\mbox{}\verb@if __name__ == "__main__" :@\\
\mbox{}\verb@    @\hbox{$\langle\,${\itshape get program options}\nobreak\ {\footnotesize \NWlink{nuweb?}{?}}$\,\rangle$}\verb@@\\
\mbox{}\verb@    @\hbox{$\langle\,${\itshape print the testpost}\nobreak\ {\footnotesize \NWlink{nuweb?}{?}}$\,\rangle$}\verb@@\\
\mbox{}\verb@@\\
\mbox{}\verb@@{\NWsep}
\end{list}
\vspace{-1.5ex}
\footnotesize
\begin{list}{}{\setlength{\itemsep}{-\parsep}\setlength{\itemindent}{-\leftmargin}}

\item{}
\end{list}
\end{minipage}\vspace{4ex}
\end{flushleft}
For now, the program just prints a mock-up of a post:

\begin{flushleft} \small
\begin{minipage}{\linewidth}\label{scrap4}\raggedright\small
\NWtarget{nuweb?}{} $\langle\,${\itshape print the testpost}\nobreak\ {\footnotesize {?}}$\,\rangle\equiv$
\vspace{-1ex}
\begin{list}{}{} \item
\mbox{}\verb@print_post(boardnum, "CEVT", "Gallup: life got better", 1, "juddism", datetime.datetime.now(), "Come on now")@\\
\mbox{}\verb@@{\NWsep}
\end{list}
\vspace{-1.5ex}
\footnotesize
\begin{list}{}{\setlength{\itemsep}{-\parsep}\setlength{\itemindent}{-\leftmargin}}
\item \NWtxtMacroRefIn\ \NWlink{nuweb?}{?}.
\item \NWtxtIdentsUsed\nobreak\  \verb@boardnum@\nobreak\ \NWlink{nuweb?}{?}, \verb@print_post@\nobreak\ \NWlink{nuweb?}{?}.
\item{}
\end{list}
\end{minipage}\vspace{4ex}
\end{flushleft}
\begin{flushleft} \small
\begin{minipage}{\linewidth}\label{scrap5}\raggedright\small
\NWtarget{nuweb?}{} $\langle\,${\itshape import modules in main program}\nobreak\ {\footnotesize {?}}$\,\rangle\equiv$
\vspace{-1ex}
\begin{list}{}{} \item
\mbox{}\verb@import datetime@\\
\mbox{}\verb@@{\NWsep}
\end{list}
\vspace{-1.5ex}
\footnotesize
\begin{list}{}{\setlength{\itemsep}{-\parsep}\setlength{\itemindent}{-\leftmargin}}
\item \NWtxtMacroRefIn\ \NWlink{nuweb?}{?}.

\item{}
\end{list}
\end{minipage}\vspace{4ex}
\end{flushleft}
\appendix

\section{How to read and translate this document}
\label{sec:translatedoc}

This document is an example of \emph{literate
  programming}~\cite{Knuth:1983:LP}. It contains the code of all sorts
of scripts and programs, combined with explaining texts. In this
document the literate programming tool \texttt{nuweb} is used, that is
currently available from Sourceforge
(URL:\url{nuweb.sourceforge.net}). The advantages of Nuweb are, that
it can be used for every programming language and scripting language, that
it can contain multiple program sources and that it is very simple.


\subsection{Read this document}
\label{sec:read}

The document contains \emph{code scraps} that are collected into
output files. An output file (e.g. \texttt{output.fil}) shows up in the text as follows:

\begin{alltt}
"output.fil" \textrm{4a \(\equiv\)}
      # output.fil
      \textrm{\(<\) a macro 4b \(>\)}
      \textrm{\(<\) another macro 4c \(>\)}
      \(\diamond\)

\end{alltt}

The above construction contains text for the file. It is labelled with
a code (in this case 4a)  The constructions between the \(<\) and
\(>\) brackets are macro's, placeholders for texts that can be found
in other places of the document. The test for a macro is found in
constructions that look like:

\begin{alltt}
\textrm{\(<\) a macro 4b \(>\) \(\equiv\)}
     This is a scrap of code inside the macro.
     It is concatenated with other scraps inside the
     macro. The concatenated scraps replace
     the invocation of the macro.

{\footnotesize\textrm Macro defined by 4b, 87e}
{\footnotesize\textrm Macro referenced in 4a}
\end{alltt}

Macro's can be defined on different places. They can contain other macro´s.

\begin{alltt}
\textrm{\(<\) a scrap 87e \(>\) \(\equiv\)}
     This is another scrap in the macro. It is
     concatenated to the text of scrap 4b.
     This scrap contains another macro:
     \textrm{\(<\) another macro 45b \(>\)}

{\footnotesize\textrm Macro defined by 4b, 87e}
{\footnotesize\textrm Macro referenced in 4a}
\end{alltt}


\subsection{Process the document}
\label{sec:processing}

The raw document is named
\verb|a_myscrapexamp.w|. Figure~\ref{fig:fileschema}
\begin{figure}[hbtp]
  \centering
  \includegraphics{fileschema.fig}
  \caption{Translation of the raw code of this document into
    printable/viewable documents and into program sources. The figure
    shows the pathways and the main files involved.}
  \label{fig:fileschema}
\end{figure}
 shows pathways to
translate it into printable/viewable documents and to extract the
program sources. Table~\ref{tab:transtools}
\begin{table}[hbtp]
  \centering
  \begin{tabular}{lll}
    \textbf{Tool} & \textbf{Source} & \textbf{Description} \\
    gawk  & \url{www.gnu.org/software/gawk/}& text-processing scripting language \\
    M4    & \url{www.gnu.org/software/m4/}& Gnu macro processor \\
    nuweb & \url{nuweb.sourceforge.net} & Literate programming tool \\
    tex   & \url{www.ctan.org} & Typesetting system \\
    tex4ht & \url{www.ctan.org} & Convert \TeX{} documents into \texttt{xml}/\texttt{html}
  \end{tabular}
  \caption{Tools to translate this document into readable code and to
    extract the program sources}
  \label{tab:transtools}
\end{table}
lists the tools that are
needed for a translation. Most of the tools (except Nuweb) are available on a
well-equipped Linux system.


\begin{flushleft} \small
\begin{minipage}{\linewidth}\label{scrap6}\raggedright\small
\NWtarget{nuweb?}{} $\langle\,${\itshape parameters in Makefile}\nobreak\ {\footnotesize {?}}$\,\rangle\equiv$
\vspace{-1ex}
\begin{list}{}{} \item
\mbox{}\verb@NUWEB=/usr/local/bin/nuweb@\\
\mbox{}\verb@@{\NWsep}
\end{list}
\vspace{-1.5ex}
\footnotesize
\begin{list}{}{\setlength{\itemsep}{-\parsep}\setlength{\itemindent}{-\leftmargin}}
\item \NWtxtMacroDefBy\ \NWlink{nuweb?}{?}\NWlink{nuweb?}{, ?}\NWlink{nuweb?}{, ?}\NWlink{nuweb?}{, ?}\NWlink{nuweb?}{, ?}\NWlink{nuweb?}{, ?}\NWlink{nuweb?}{, ?}.
\item \NWtxtMacroRefIn\ \NWlink{nuweb?}{?}.
\item \NWtxtIdentsUsed\nobreak\  \verb@nuweb@\nobreak\ \NWlink{nuweb?}{?}.
\item{}
\end{list}
\end{minipage}\vspace{4ex}
\end{flushleft}
\subsection{Translate and run}
\label{sec:transrun}

This chapter assembles the Makefile for this project.

\begin{flushleft} \small
\begin{minipage}{\linewidth}\label{scrap7}\raggedright\small
\NWtarget{nuweb?}{} \verb@"Makefile"@\nobreak\ {\footnotesize {?}}$\equiv$
\vspace{-1ex}
\begin{list}{}{} \item
\mbox{}\verb@@\hbox{$\langle\,${\itshape default target}\nobreak\ {\footnotesize \NWlink{nuweb?}{?}}$\,\rangle$}\verb@@\\
\mbox{}\verb@@\\
\mbox{}\verb@@\hbox{$\langle\,${\itshape parameters in Makefile}\nobreak\ {\footnotesize \NWlink{nuweb?}{?}, \ldots\ }$\,\rangle$}\verb@ @\\
\mbox{}\verb@@\\
\mbox{}\verb@@\hbox{$\langle\,${\itshape impliciete make regels}\nobreak\ {\footnotesize \NWlink{nuweb?}{?}, \ldots\ }$\,\rangle$}\verb@@\\
\mbox{}\verb@@\hbox{$\langle\,${\itshape expliciete make regels}\nobreak\ {\footnotesize \NWlink{nuweb?}{?}, \ldots\ }$\,\rangle$}\verb@@\\
\mbox{}\verb@@\hbox{$\langle\,${\itshape make targets}\nobreak\ {\footnotesize \NWlink{nuweb?}{?}, \ldots\ }$\,\rangle$}\verb@@\\
\mbox{}\verb@@{\NWsep}
\end{list}
\vspace{-1.5ex}
\footnotesize
\begin{list}{}{\setlength{\itemsep}{-\parsep}\setlength{\itemindent}{-\leftmargin}}

\item{}
\end{list}
\end{minipage}\vspace{4ex}
\end{flushleft}
The default target of make is \verb|all|.

\begin{flushleft} \small
\begin{minipage}{\linewidth}\label{scrap8}\raggedright\small
\NWtarget{nuweb?}{} $\langle\,${\itshape default target}\nobreak\ {\footnotesize {?}}$\,\rangle\equiv$
\vspace{-1ex}
\begin{list}{}{} \item
\mbox{}\verb@all : @\hbox{$\langle\,${\itshape all targets}\nobreak\ {\footnotesize \NWlink{nuweb?}{?}}$\,\rangle$}\verb@@\\
\mbox{}\verb@.PHONY : all@\\
\mbox{}\verb@@\\
\mbox{}\verb@@{\NWsep}
\end{list}
\vspace{-1.5ex}
\footnotesize
\begin{list}{}{\setlength{\itemsep}{-\parsep}\setlength{\itemindent}{-\leftmargin}}
\item \NWtxtMacroRefIn\ \NWlink{nuweb?}{?}.
\item \NWtxtIdentsDefed\nobreak\  \verb@all@\nobreak\ \NWtxtIdentsNotUsed, \verb@PHONY@\nobreak\ \NWlink{nuweb?}{?}.
\item{}
\end{list}
\end{minipage}\vspace{4ex}
\end{flushleft}
One of the targets is certainly the \textsc{pdf} version of this
document.

\begin{flushleft} \small
\begin{minipage}{\linewidth}\label{scrap9}\raggedright\small
\NWtarget{nuweb?}{} $\langle\,${\itshape all targets}\nobreak\ {\footnotesize {?}}$\,\rangle\equiv$
\vspace{-1ex}
\begin{list}{}{} \item
\mbox{}\verb@myscrapexamp.pdf@{\NWsep}
\end{list}
\vspace{-1.5ex}
\footnotesize
\begin{list}{}{\setlength{\itemsep}{-\parsep}\setlength{\itemindent}{-\leftmargin}}
\item \NWtxtMacroRefIn\ \NWlink{nuweb?}{?}.
\item \NWtxtIdentsUsed\nobreak\  \verb@pdf@\nobreak\ \NWlink{nuweb?}{?}.
\item{}
\end{list}
\end{minipage}\vspace{4ex}
\end{flushleft}
We use many suffixes that were not known by the C-programmers who
constructed the \texttt{make} utility. Add these suffixes to the list.

\begin{flushleft} \small
\begin{minipage}{\linewidth}\label{scrap10}\raggedright\small
\NWtarget{nuweb?}{} $\langle\,${\itshape parameters in Makefile}\nobreak\ {\footnotesize {?}}$\,\rangle\equiv$
\vspace{-1ex}
\begin{list}{}{} \item
\mbox{}\verb@.SUFFIXES: .pdf .w .tex .html .aux .log .php@\\
\mbox{}\verb@@\\
\mbox{}\verb@@{\NWsep}
\end{list}
\vspace{-1.5ex}
\footnotesize
\begin{list}{}{\setlength{\itemsep}{-\parsep}\setlength{\itemindent}{-\leftmargin}}
\item \NWtxtMacroDefBy\ \NWlink{nuweb?}{?}\NWlink{nuweb?}{, ?}\NWlink{nuweb?}{, ?}\NWlink{nuweb?}{, ?}\NWlink{nuweb?}{, ?}\NWlink{nuweb?}{, ?}\NWlink{nuweb?}{, ?}.
\item \NWtxtMacroRefIn\ \NWlink{nuweb?}{?}.
\item \NWtxtIdentsDefed\nobreak\  \verb@SUFFIXES@\nobreak\ \NWtxtIdentsNotUsed.\item \NWtxtIdentsUsed\nobreak\  \verb@pdf@\nobreak\ \NWlink{nuweb?}{?}.
\item{}
\end{list}
\end{minipage}\vspace{4ex}
\end{flushleft}
\subsection{Pre-processing}
\label{sec:pre-processing}

To make usable things from the raw input \verb|a_myscrapexamp.w|, do the following:

\begin{enumerate}
\item Process \verb|$| characters.
\item Run the m4 pre-processor.
\item Run nuweb.
\end{enumerate}

This results in a \LaTeX{} file, that can be converted into a \pdf{}
or a \HTML{} document, and in the program sources and scripts.

\subsubsection{Process `dollar' characters }
\label{sec:procdollars}

Many ``intelligent'' \TeX{} editors (e.g.\ the auctex utility of
Emacs) handle \verb|$| characters as special, to switch into
mathematics mode. This is irritating in program texts, that often
contain \verb|$| characters as well. Therefore, we make a stub, that
translates the two-character sequence \verb|\$| into the single
\verb|$| character.


\begin{flushleft} \small
\begin{minipage}{\linewidth}\label{scrap11}\raggedright\small
\NWtarget{nuweb?}{} $\langle\,${\itshape expliciete make regels}\nobreak\ {\footnotesize {?}}$\,\rangle\equiv$
\vspace{-1ex}
\begin{list}{}{} \item
\mbox{}\verb@m4_myscrapexamp.w : a_myscrapexamp.w@\\
\mbox{}\verb@        gawk '{if(match($$0, "@{\tt @}\verb@%")) {printf("%s", substr($$0,1,RSTART-1))} else print}' a_myscrapexamp.w \@\\
\mbox{}\verb@          | gawk '{gsub(/[\\][\$$]/, "$$");print}'  > m4_myscrapexamp.w@\\
\mbox{}\verb@@\\
\mbox{}\verb@@{\NWsep}
\end{list}
\vspace{-1.5ex}
\footnotesize
\begin{list}{}{\setlength{\itemsep}{-\parsep}\setlength{\itemindent}{-\leftmargin}}
\item \NWtxtMacroDefBy\ \NWlink{nuweb?}{?}\NWlink{nuweb?}{, ?}\NWlink{nuweb?}{, ?}\NWlink{nuweb?}{, ?}\NWlink{nuweb?}{, ?}\NWlink{nuweb?}{, ?}\NWlink{nuweb?}{, ?}\NWlink{nuweb?}{, ?}.
\item \NWtxtMacroRefIn\ \NWlink{nuweb?}{?}.
\item \NWtxtIdentsUsed\nobreak\  \verb@print@\nobreak\ \NWlink{nuweb?}{?}.
\item{}
\end{list}
\end{minipage}\vspace{4ex}
\end{flushleft}
\subsubsection{Run the M4 pre-processor}
\label{sec:run_M4}

\begin{flushleft} \small
\begin{minipage}{\linewidth}\label{scrap12}\raggedright\small
\NWtarget{nuweb?}{} $\langle\,${\itshape expliciete make regels}\nobreak\ {\footnotesize {?}}$\,\rangle\equiv$
\vspace{-1ex}
\begin{list}{}{} \item
\mbox{}\verb@myscrapexamp.w : m4_myscrapexamp.w@\\
\mbox{}\verb@        m4 -P m4_myscrapexamp.w > myscrapexamp.w@\\
\mbox{}\verb@@\\
\mbox{}\verb@@{\NWsep}
\end{list}
\vspace{-1.5ex}
\footnotesize
\begin{list}{}{\setlength{\itemsep}{-\parsep}\setlength{\itemindent}{-\leftmargin}}
\item \NWtxtMacroDefBy\ \NWlink{nuweb?}{?}\NWlink{nuweb?}{, ?}\NWlink{nuweb?}{, ?}\NWlink{nuweb?}{, ?}\NWlink{nuweb?}{, ?}\NWlink{nuweb?}{, ?}\NWlink{nuweb?}{, ?}\NWlink{nuweb?}{, ?}.
\item \NWtxtMacroRefIn\ \NWlink{nuweb?}{?}.

\item{}
\end{list}
\end{minipage}\vspace{4ex}
\end{flushleft}
\subsection{Typeset this document}
\label{sec:typeset}

Enable the following:
\begin{enumerate}
\item Create a \pdf{} document.
\item Print the typeset document.
\item View the typeset document with a viewer.
\item Create a \HTML document.
\end{enumerate}

In the three items, a typeset \pdf{} document is required or it is the
requirement itself.




\subsubsection{Figures}
\label{sec:figures}

This document contains figures that have been made by
\texttt{xfig}. Post-process the figures to enable inclusion in this
document.

The list of figures to be included:

\begin{flushleft} \small
\begin{minipage}{\linewidth}\label{scrap13}\raggedright\small
\NWtarget{nuweb?}{} $\langle\,${\itshape parameters in Makefile}\nobreak\ {\footnotesize {?}}$\,\rangle\equiv$
\vspace{-1ex}
\begin{list}{}{} \item
\mbox{}\verb@FIGFILES=fileschema@\\
\mbox{}\verb@@\\
\mbox{}\verb@@{\NWsep}
\end{list}
\vspace{-1.5ex}
\footnotesize
\begin{list}{}{\setlength{\itemsep}{-\parsep}\setlength{\itemindent}{-\leftmargin}}
\item \NWtxtMacroDefBy\ \NWlink{nuweb?}{?}\NWlink{nuweb?}{, ?}\NWlink{nuweb?}{, ?}\NWlink{nuweb?}{, ?}\NWlink{nuweb?}{, ?}\NWlink{nuweb?}{, ?}\NWlink{nuweb?}{, ?}.
\item \NWtxtMacroRefIn\ \NWlink{nuweb?}{?}.
\item \NWtxtIdentsDefed\nobreak\  \verb@FIGFILES@\nobreak\ \NWlink{nuweb?}{?}\NWlink{nuweb?}{, ?}.
\item{}
\end{list}
\end{minipage}\vspace{4ex}
\end{flushleft}
We use the package \texttt{figlatex} to include the pictures. This
package expects two files with extensions \verb|.pdftex| and
\verb|.pdftex_t| for \texttt{pdflatex} and two files with extensions \verb|.pstex| and
\verb|.pstex_t| for the \texttt{latex}/\texttt{dvips}
combination. Probably tex4ht uses the latter two formats too.

Make lists of the graphical files that have to be present for
latex/pdflatex:

\begin{flushleft} \small
\begin{minipage}{\linewidth}\label{scrap14}\raggedright\small
\NWtarget{nuweb?}{} $\langle\,${\itshape parameters in Makefile}\nobreak\ {\footnotesize {?}}$\,\rangle\equiv$
\vspace{-1ex}
\begin{list}{}{} \item
\mbox{}\verb@FIGFILENAMES=$(foreach fil,$(FIGFILES), $(fil).fig)@\\
\mbox{}\verb@PDFT_NAMES=$(foreach fil,$(FIGFILES), $(fil).pdftex_t)@\\
\mbox{}\verb@PDF_FIG_NAMES=$(foreach fil,$(FIGFILES), $(fil).pdftex)@\\
\mbox{}\verb@PST_NAMES=$(foreach fil,$(FIGFILES), $(fil).pstex_t)@\\
\mbox{}\verb@PS_FIG_NAMES=$(foreach fil,$(FIGFILES), $(fil).pstex)@\\
\mbox{}\verb@@\\
\mbox{}\verb@@{\NWsep}
\end{list}
\vspace{-1.5ex}
\footnotesize
\begin{list}{}{\setlength{\itemsep}{-\parsep}\setlength{\itemindent}{-\leftmargin}}
\item \NWtxtMacroDefBy\ \NWlink{nuweb?}{?}\NWlink{nuweb?}{, ?}\NWlink{nuweb?}{, ?}\NWlink{nuweb?}{, ?}\NWlink{nuweb?}{, ?}\NWlink{nuweb?}{, ?}\NWlink{nuweb?}{, ?}.
\item \NWtxtMacroRefIn\ \NWlink{nuweb?}{?}.
\item \NWtxtIdentsDefed\nobreak\  \verb@FIGFILENAMES@\nobreak\ \NWtxtIdentsNotUsed, \verb@PDFT_NAMES@\nobreak\ \NWlink{nuweb?}{?}, \verb@PDF_FIG_NAMES@\nobreak\ \NWlink{nuweb?}{?}, \verb@PST_NAMES@\nobreak\ \NWtxtIdentsNotUsed, \verb@PS_FIG_NAMES@\nobreak\ \NWtxtIdentsNotUsed.\item \NWtxtIdentsUsed\nobreak\  \verb@FIGFILES@\nobreak\ \NWlink{nuweb?}{?}.
\item{}
\end{list}
\end{minipage}\vspace{4ex}
\end{flushleft}
Create
the graph files with program \verb|fig2dev|:

\begin{flushleft} \small
\begin{minipage}{\linewidth}\label{scrap15}\raggedright\small
\NWtarget{nuweb?}{} $\langle\,${\itshape impliciete make regels}\nobreak\ {\footnotesize {?}}$\,\rangle\equiv$
\vspace{-1ex}
\begin{list}{}{} \item
\mbox{}\verb@%.eps: %.fig@\\
\mbox{}\verb@        fig2dev -L eps $< > $@{\tt @}\verb@@\\
\mbox{}\verb@@\\
\mbox{}\verb@%.pstex: %.fig@\\
\mbox{}\verb@        fig2dev -L pstex $< > $@{\tt @}\verb@@\\
\mbox{}\verb@@\\
\mbox{}\verb@.PRECIOUS : %.pstex@\\
\mbox{}\verb@%.pstex_t: %.fig %.pstex@\\
\mbox{}\verb@        fig2dev -L pstex_t -p $*.pstex $< > $@{\tt @}\verb@@\\
\mbox{}\verb@@\\
\mbox{}\verb@%.pdftex: %.fig@\\
\mbox{}\verb@        fig2dev -L pdftex $< > $@{\tt @}\verb@@\\
\mbox{}\verb@@\\
\mbox{}\verb@.PRECIOUS : %.pdftex@\\
\mbox{}\verb@%.pdftex_t: %.fig %.pstex@\\
\mbox{}\verb@        fig2dev -L pdftex_t -p $*.pdftex $< > $@{\tt @}\verb@@\\
\mbox{}\verb@@\\
\mbox{}\verb@@{\NWsep}
\end{list}
\vspace{-1.5ex}
\footnotesize
\begin{list}{}{\setlength{\itemsep}{-\parsep}\setlength{\itemindent}{-\leftmargin}}
\item \NWtxtMacroDefBy\ \NWlink{nuweb?}{?}\NWlink{nuweb?}{, ?}\NWlink{nuweb?}{, ?}.
\item \NWtxtMacroRefIn\ \NWlink{nuweb?}{?}.
\item \NWtxtIdentsDefed\nobreak\  \verb@fig2dev@\nobreak\ \NWtxtIdentsNotUsed.
\item{}
\end{list}
\end{minipage}\vspace{4ex}
\end{flushleft}
\subsubsection{Bibliography}
\label{sec:bbliography}

To keep this document portable, create a portable bibliography
file. It works as follows: This document refers in the
\texttt|bibliography| statement to the local \verb|bib|-file
\verb|myscrapexamp.bib|. To create this file, copy the auxiliary file
to another file \verb|auxfil.aux|, but replace the argument of the
command \verb|\bibdata{myscrapexamp}| to the names of the bibliography
files that contain the actual references (they should exist on the
computer on which you try this). This procedure should only be
performed on the computer of the author. Therefore, it is dependent of
a binary file on his computer.


\begin{flushleft} \small
\begin{minipage}{\linewidth}\label{scrap16}\raggedright\small
\NWtarget{nuweb?}{} $\langle\,${\itshape expliciete make regels}\nobreak\ {\footnotesize {?}}$\,\rangle\equiv$
\vspace{-1ex}
\begin{list}{}{} \item
\mbox{}\verb@bibfile : myscrapexamp.aux /home/paul/bin/mkportbib@\\
\mbox{}\verb@        /home/paul/bin/mkportbib myscrapexamp litprog@\\
\mbox{}\verb@@\\
\mbox{}\verb@.PHONY : bibfile@\\
\mbox{}\verb@@{\NWsep}
\end{list}
\vspace{-1.5ex}
\footnotesize
\begin{list}{}{\setlength{\itemsep}{-\parsep}\setlength{\itemindent}{-\leftmargin}}
\item \NWtxtMacroDefBy\ \NWlink{nuweb?}{?}\NWlink{nuweb?}{, ?}\NWlink{nuweb?}{, ?}\NWlink{nuweb?}{, ?}\NWlink{nuweb?}{, ?}\NWlink{nuweb?}{, ?}\NWlink{nuweb?}{, ?}\NWlink{nuweb?}{, ?}.
\item \NWtxtMacroRefIn\ \NWlink{nuweb?}{?}.
\item \NWtxtIdentsUsed\nobreak\  \verb@PHONY@\nobreak\ \NWlink{nuweb?}{?}.
\item{}
\end{list}
\end{minipage}\vspace{4ex}
\end{flushleft}
\subsubsection{Create a printable/viewable document}
\label{sec:createpdf}

Make a \pdf{} document for printing and viewing.

\begin{flushleft} \small
\begin{minipage}{\linewidth}\label{scrap17}\raggedright\small
\NWtarget{nuweb?}{} $\langle\,${\itshape make targets}\nobreak\ {\footnotesize {?}}$\,\rangle\equiv$
\vspace{-1ex}
\begin{list}{}{} \item
\mbox{}\verb@pdf : myscrapexamp.pdf@\\
\mbox{}\verb@@\\
\mbox{}\verb@print : myscrapexamp.pdf@\\
\mbox{}\verb@        lpr myscrapexamp.pdf@\\
\mbox{}\verb@@\\
\mbox{}\verb@view : myscrapexamp.pdf@\\
\mbox{}\verb@        evince myscrapexamp.pdf@\\
\mbox{}\verb@@\\
\mbox{}\verb@@{\NWsep}
\end{list}
\vspace{-1.5ex}
\footnotesize
\begin{list}{}{\setlength{\itemsep}{-\parsep}\setlength{\itemindent}{-\leftmargin}}
\item \NWtxtMacroDefBy\ \NWlink{nuweb?}{?}\NWlink{nuweb?}{, ?}\NWlink{nuweb?}{, ?}\NWlink{nuweb?}{, ?}.
\item \NWtxtMacroRefIn\ \NWlink{nuweb?}{?}.
\item \NWtxtIdentsDefed\nobreak\  \verb@pdf@\nobreak\ \NWlink{nuweb?}{?}\NWlink{nuweb?}{, ?}\NWlink{nuweb?}{, ?}, \verb@print@\nobreak\ \NWlink{nuweb?}{?}\NWlink{nuweb?}{, ?}, \verb@view@\nobreak\ \NWtxtIdentsNotUsed.
\item{}
\end{list}
\end{minipage}\vspace{4ex}
\end{flushleft}
Create the \pdf{} document. This may involve multiple runs of nuweb,
the \LaTeX{} processor and the bib\TeX{} processor, and depends on the
state of the \verb|aux| file that the \LaTeX{} processor creates as a
by-product. Therefore, this is performed in a separate script,
\verb|w2pdf|.

\paragraph{The w2pdf script}
\label{sec:w2pdf}

The three processors nuweb, \LaTeX{} and bib\TeX{} are
intertwined. \LaTeX{} and bib\TeX{} create parameters or change the
value of parameters, and write them in an auxiliary file. The other
processors may need those values to produce the correct output. The
\LaTeX{} processor may even need the parameters in a second
run. Therefore, consider the creation of the (\pdf) document finished
when none of the processors causes the auxiliary file to change. This
is performed by a shell script \verb|w2pdf|.




Note, that in the following \texttt{make} construct, the implicit rule
\verb|.w.pdf| is not used. It turned out, that make did not calculate
the dependencies correctly when I did use this rule.

\begin{flushleft} \small
\begin{minipage}{\linewidth}\label{scrap18}\raggedright\small
\NWtarget{nuweb?}{} $\langle\,${\itshape impliciete make regels}\nobreak\ {\footnotesize {?}}$\,\rangle\equiv$
\vspace{-1ex}
\begin{list}{}{} \item
\mbox{}\verb@%.pdf : %.w $(W2PDF)  $(PDF_FIG_NAMES) $(PDFT_NAMES)@\\
\mbox{}\verb@        chmod 775 $(W2PDF)@\\
\mbox{}\verb@        $(W2PDF) $*@\\
\mbox{}\verb@@\\
\mbox{}\verb@@{\NWsep}
\end{list}
\vspace{-1.5ex}
\footnotesize
\begin{list}{}{\setlength{\itemsep}{-\parsep}\setlength{\itemindent}{-\leftmargin}}
\item \NWtxtMacroDefBy\ \NWlink{nuweb?}{?}\NWlink{nuweb?}{, ?}\NWlink{nuweb?}{, ?}.
\item \NWtxtMacroRefIn\ \NWlink{nuweb?}{?}.
\item \NWtxtIdentsUsed\nobreak\  \verb@pdf@\nobreak\ \NWlink{nuweb?}{?}, \verb@PDFT_NAMES@\nobreak\ \NWlink{nuweb?}{?}, \verb@PDF_FIG_NAMES@\nobreak\ \NWlink{nuweb?}{?}.
\item{}
\end{list}
\end{minipage}\vspace{4ex}
\end{flushleft}
The following is an ugly fix of an unsolved problem. Currently I
develop this thing, while it resides on a remote computer that is
connected via the \verb|sshfs| filesystem. On my home computer I
cannot run executables on this system, but on my work-computer I
can. Therefore, place the following script on a local directory.

\begin{flushleft} \small
\begin{minipage}{\linewidth}\label{scrap19}\raggedright\small
\NWtarget{nuweb?}{} $\langle\,${\itshape parameters in Makefile}\nobreak\ {\footnotesize {?}}$\,\rangle\equiv$
\vspace{-1ex}
\begin{list}{}{} \item
\mbox{}\verb@W2PDF=../nuweb/bin/w2pdf@\\
\mbox{}\verb@@{\NWsep}
\end{list}
\vspace{-1.5ex}
\footnotesize
\begin{list}{}{\setlength{\itemsep}{-\parsep}\setlength{\itemindent}{-\leftmargin}}
\item \NWtxtMacroDefBy\ \NWlink{nuweb?}{?}\NWlink{nuweb?}{, ?}\NWlink{nuweb?}{, ?}\NWlink{nuweb?}{, ?}\NWlink{nuweb?}{, ?}\NWlink{nuweb?}{, ?}\NWlink{nuweb?}{, ?}.
\item \NWtxtMacroRefIn\ \NWlink{nuweb?}{?}.
\item \NWtxtIdentsUsed\nobreak\  \verb@nuweb@\nobreak\ \NWlink{nuweb?}{?}.
\item{}
\end{list}
\end{minipage}\vspace{4ex}
\end{flushleft}
\begin{flushleft} \small
\begin{minipage}{\linewidth}\label{scrap20}\raggedright\small
\NWtarget{nuweb?}{} $\langle\,${\itshape directories to create}\nobreak\ {\footnotesize {?}}$\,\rangle\equiv$
\vspace{-1ex}
\begin{list}{}{} \item
\mbox{}\verb@../nuweb/bin @{\NWsep}
\end{list}
\vspace{-1.5ex}
\footnotesize
\begin{list}{}{\setlength{\itemsep}{-\parsep}\setlength{\itemindent}{-\leftmargin}}
\item \NWtxtMacroRefIn\ \NWlink{nuweb?}{?}.
\item \NWtxtIdentsUsed\nobreak\  \verb@nuweb@\nobreak\ \NWlink{nuweb?}{?}.
\item{}
\end{list}
\end{minipage}\vspace{4ex}
\end{flushleft}
\begin{flushleft} \small
\begin{minipage}{\linewidth}\label{scrap21}\raggedright\small
\NWtarget{nuweb?}{} $\langle\,${\itshape expliciete make regels}\nobreak\ {\footnotesize {?}}$\,\rangle\equiv$
\vspace{-1ex}
\begin{list}{}{} \item
\mbox{}\verb@$(W2PDF) : myscrapexamp.w@\\
\mbox{}\verb@        $(NUWEB) myscrapexamp.w@\\
\mbox{}\verb@@{\NWsep}
\end{list}
\vspace{-1.5ex}
\footnotesize
\begin{list}{}{\setlength{\itemsep}{-\parsep}\setlength{\itemindent}{-\leftmargin}}
\item \NWtxtMacroDefBy\ \NWlink{nuweb?}{?}\NWlink{nuweb?}{, ?}\NWlink{nuweb?}{, ?}\NWlink{nuweb?}{, ?}\NWlink{nuweb?}{, ?}\NWlink{nuweb?}{, ?}\NWlink{nuweb?}{, ?}\NWlink{nuweb?}{, ?}.
\item \NWtxtMacroRefIn\ \NWlink{nuweb?}{?}.

\item{}
\end{list}
\end{minipage}\vspace{4ex}
\end{flushleft}
\begin{flushleft} \small
\begin{minipage}{\linewidth}\label{scrap22}\raggedright\small
\NWtarget{nuweb?}{} \verb@"../nuweb/bin/w2pdf"@\nobreak\ {\footnotesize {?}}$\equiv$
\vspace{-1ex}
\begin{list}{}{} \item
\mbox{}\verb@#!/bin/bash@\\
\mbox{}\verb@# w2pdf -- compile a nuweb file@\\
\mbox{}\verb@# usage: w2pdf [filename]@\\
\mbox{}\verb@# 20160907 at 0930h: Generated by nuweb from a_myscrapexamp.w@\\
\mbox{}\verb@NUWEB=/usr/local/bin/nuweb@\\
\mbox{}\verb@LATEXCOMPILER=pdflatex@\\
\mbox{}\verb@@\hbox{$\langle\,${\itshape filenames in nuweb compile script}\nobreak\ {\footnotesize \NWlink{nuweb?}{?}}$\,\rangle$}\verb@@\\
\mbox{}\verb@@\hbox{$\langle\,${\itshape compile nuweb}\nobreak\ {\footnotesize \NWlink{nuweb?}{?}}$\,\rangle$}\verb@@\\
\mbox{}\verb@@\\
\mbox{}\verb@@{\NWsep}
\end{list}
\vspace{-1.5ex}
\footnotesize
\begin{list}{}{\setlength{\itemsep}{-\parsep}\setlength{\itemindent}{-\leftmargin}}
\item \NWtxtIdentsUsed\nobreak\  \verb@nuweb@\nobreak\ \NWlink{nuweb?}{?}.
\item{}
\end{list}
\end{minipage}\vspace{4ex}
\end{flushleft}
The script retains a copy of the latest version of the auxiliary file.
Then it runs the four processors nuweb, \LaTeX{}, MakeIndex and bib\TeX{}, until
they do not change the auxiliary file or the index. 

\begin{flushleft} \small
\begin{minipage}{\linewidth}\label{scrap23}\raggedright\small
\NWtarget{nuweb?}{} $\langle\,${\itshape compile nuweb}\nobreak\ {\footnotesize {?}}$\,\rangle\equiv$
\vspace{-1ex}
\begin{list}{}{} \item
\mbox{}\verb@NUWEB=m4_nuweb@\\
\mbox{}\verb@@\hbox{$\langle\,${\itshape run the processors until the aux file remains unchanged}\nobreak\ {\footnotesize \NWlink{nuweb?}{?}}$\,\rangle$}\verb@@\\
\mbox{}\verb@@\hbox{$\langle\,${\itshape remove the copy of the aux file}\nobreak\ {\footnotesize \NWlink{nuweb?}{?}}$\,\rangle$}\verb@@\\
\mbox{}\verb@@{\NWsep}
\end{list}
\vspace{-1.5ex}
\footnotesize
\begin{list}{}{\setlength{\itemsep}{-\parsep}\setlength{\itemindent}{-\leftmargin}}
\item \NWtxtMacroRefIn\ \NWlink{nuweb?}{?}.

\item{}
\end{list}
\end{minipage}\vspace{4ex}
\end{flushleft}
The user provides the name of the nuweb file as argument. Strip the
extension (e.g.\ \verb|.w|) from the filename and create the names of
the \LaTeX{} file (ends with \verb|.tex|), the auxiliary file (ends
with \verb|.aux|) and the copy of the auxiliary file (add \verb|old.|
as a prefix to the auxiliary filename).

\begin{flushleft} \small
\begin{minipage}{\linewidth}\label{scrap24}\raggedright\small
\NWtarget{nuweb?}{} $\langle\,${\itshape filenames in nuweb compile script}\nobreak\ {\footnotesize {?}}$\,\rangle\equiv$
\vspace{-1ex}
\begin{list}{}{} \item
\mbox{}\verb@nufil=$1@\\
\mbox{}\verb@trunk=${1%%.*}@\\
\mbox{}\verb@texfil=${trunk}.tex@\\
\mbox{}\verb@auxfil=${trunk}.aux@\\
\mbox{}\verb@oldaux=old.${trunk}.aux@\\
\mbox{}\verb@indexfil=${trunk}.idx@\\
\mbox{}\verb@oldindexfil=old.${trunk}.idx@\\
\mbox{}\verb@@{\NWsep}
\end{list}
\vspace{-1.5ex}
\footnotesize
\begin{list}{}{\setlength{\itemsep}{-\parsep}\setlength{\itemindent}{-\leftmargin}}
\item \NWtxtMacroRefIn\ \NWlink{nuweb?}{?}.
\item \NWtxtIdentsDefed\nobreak\  \verb@auxfil@\nobreak\ \NWlink{nuweb?}{?}\NWlink{nuweb?}{, ?}\NWlink{nuweb?}{, ?}, \verb@indexfil@\nobreak\ \NWlink{nuweb?}{?}\NWlink{nuweb?}{, ?}, \verb@nufil@\nobreak\ \NWlink{nuweb?}{?}\NWlink{nuweb?}{, ?}\NWlink{nuweb?}{, ?}, \verb@oldaux@\nobreak\ \NWlink{nuweb?}{?}\NWlink{nuweb?}{, ?}\NWlink{nuweb?}{, ?}\NWlink{nuweb?}{, ?}, \verb@oldindexfil@\nobreak\ \NWlink{nuweb?}{?}\NWlink{nuweb?}{, ?}, \verb@texfil@\nobreak\ \NWlink{nuweb?}{?}\NWlink{nuweb?}{, ?}\NWlink{nuweb?}{, ?}, \verb@trunk@\nobreak\ \NWlink{nuweb?}{?}\NWlink{nuweb?}{, ?}\NWlink{nuweb?}{, ?}\NWlink{nuweb?}{, ?}.
\item{}
\end{list}
\end{minipage}\vspace{4ex}
\end{flushleft}
Remove the old copy if it is no longer needed.
\begin{flushleft} \small
\begin{minipage}{\linewidth}\label{scrap25}\raggedright\small
\NWtarget{nuweb?}{} $\langle\,${\itshape remove the copy of the aux file}\nobreak\ {\footnotesize {?}}$\,\rangle\equiv$
\vspace{-1ex}
\begin{list}{}{} \item
\mbox{}\verb@rm $oldaux@\\
\mbox{}\verb@@{\NWsep}
\end{list}
\vspace{-1.5ex}
\footnotesize
\begin{list}{}{\setlength{\itemsep}{-\parsep}\setlength{\itemindent}{-\leftmargin}}
\item \NWtxtMacroRefIn\ \NWlink{nuweb?}{?}\NWlink{nuweb?}{, ?}.
\item \NWtxtIdentsUsed\nobreak\  \verb@oldaux@\nobreak\ \NWlink{nuweb?}{?}\NWlink{nuweb?}{, ?}.
\item{}
\end{list}
\end{minipage}\vspace{4ex}
\end{flushleft}
Run the three processors. Do not use the option \verb|-o| (to suppres
generation of program sources) for nuweb,  because \verb|w2pdf| must
be kept up to date as well.

\begin{flushleft} \small
\begin{minipage}{\linewidth}\label{scrap26}\raggedright\small
\NWtarget{nuweb?}{} $\langle\,${\itshape run the three processors}\nobreak\ {\footnotesize {?}}$\,\rangle\equiv$
\vspace{-1ex}
\begin{list}{}{} \item
\mbox{}\verb@$NUWEB $nufil@\\
\mbox{}\verb@$LATEXCOMPILER $texfil@\\
\mbox{}\verb@makeindex $trunk@\\
\mbox{}\verb@bibtex $trunk@\\
\mbox{}\verb@@{\NWsep}
\end{list}
\vspace{-1.5ex}
\footnotesize
\begin{list}{}{\setlength{\itemsep}{-\parsep}\setlength{\itemindent}{-\leftmargin}}
\item \NWtxtMacroRefIn\ \NWlink{nuweb?}{?}.
\item \NWtxtIdentsDefed\nobreak\  \verb@bibtex@\nobreak\ \NWlink{nuweb?}{?}\NWlink{nuweb?}{, ?}, \verb@makeindex@\nobreak\ \NWlink{nuweb?}{?}\NWlink{nuweb?}{, ?}, \verb@nuweb@\nobreak\ \NWlink{nuweb?}{?}\NWlink{nuweb?}{, ?}\NWlink{nuweb?}{, ?}\NWlink{nuweb?}{, ?}\NWlink{nuweb?}{, ?}.\item \NWtxtIdentsUsed\nobreak\  \verb@nufil@\nobreak\ \NWlink{nuweb?}{?}\NWlink{nuweb?}{, ?}, \verb@texfil@\nobreak\ \NWlink{nuweb?}{?}\NWlink{nuweb?}{, ?}, \verb@trunk@\nobreak\ \NWlink{nuweb?}{?}\NWlink{nuweb?}{, ?}.
\item{}
\end{list}
\end{minipage}\vspace{4ex}
\end{flushleft}
Repeat to copy the auxiliary file and the index file  and run the processors until the
auxiliary file and the index file are equal to their copies.
 However, since I have not yet been able to test the \verb|aux|
file and the \verb|idx| in the same test statement, currently only the
\verb|aux| file is tested.

It turns out, that sometimes a strange loop occurs in which the
\verb|aux| file will keep to change. Therefore, with a counter we
prevent the loop to occur more than 10 times.

\begin{flushleft} \small
\begin{minipage}{\linewidth}\label{scrap27}\raggedright\small
\NWtarget{nuweb?}{} $\langle\,${\itshape run the processors until the aux file remains unchanged}\nobreak\ {\footnotesize {?}}$\,\rangle\equiv$
\vspace{-1ex}
\begin{list}{}{} \item
\mbox{}\verb@LOOPCOUNTER=0@\\
\mbox{}\verb@while@\\
\mbox{}\verb@  ! cmp -s $auxfil $oldaux @\\
\mbox{}\verb@do@\\
\mbox{}\verb@  if [ -e $auxfil ]@\\
\mbox{}\verb@  then@\\
\mbox{}\verb@   cp $auxfil $oldaux@\\
\mbox{}\verb@  fi@\\
\mbox{}\verb@  if [ -e $indexfil ]@\\
\mbox{}\verb@  then@\\
\mbox{}\verb@   cp $indexfil $oldindexfil@\\
\mbox{}\verb@  fi@\\
\mbox{}\verb@  @\hbox{$\langle\,${\itshape run the three processors}\nobreak\ {\footnotesize \NWlink{nuweb?}{?}}$\,\rangle$}\verb@@\\
\mbox{}\verb@  if [ $LOOPCOUNTER -ge 10 ]@\\
\mbox{}\verb@  then@\\
\mbox{}\verb@    cp $auxfil $oldaux@\\
\mbox{}\verb@  fi;@\\
\mbox{}\verb@done@\\
\mbox{}\verb@@{\NWsep}
\end{list}
\vspace{-1.5ex}
\footnotesize
\begin{list}{}{\setlength{\itemsep}{-\parsep}\setlength{\itemindent}{-\leftmargin}}
\item \NWtxtMacroRefIn\ \NWlink{nuweb?}{?}.
\item \NWtxtIdentsUsed\nobreak\  \verb@auxfil@\nobreak\ \NWlink{nuweb?}{?}\NWlink{nuweb?}{, ?}, \verb@indexfil@\nobreak\ \NWlink{nuweb?}{?}, \verb@oldaux@\nobreak\ \NWlink{nuweb?}{?}\NWlink{nuweb?}{, ?}, \verb@oldindexfil@\nobreak\ \NWlink{nuweb?}{?}.
\item{}
\end{list}
\end{minipage}\vspace{4ex}
\end{flushleft}
\subsubsection{Create HTML files}
\label{sec:createhtml}

\textsc{Html} is easier to read on-line than a \pdf{} document that
was made for printing. We use \verb|tex4ht| to generate \HTML{}
code. An advantage of this system is, that we can include figures
in the same way as we do for \verb|pdflatex|.

Nuweb creates a \LaTeX{} file that is suitable
for \verb|latex2html| if the source file has \verb|.hw| as suffix instead of
\verb|.w|. However, this feature is not compatible with tex4ht.

Make html file:

\begin{flushleft} \small
\begin{minipage}{\linewidth}\label{scrap28}\raggedright\small
\NWtarget{nuweb?}{} $\langle\,${\itshape make targets}\nobreak\ {\footnotesize {?}}$\,\rangle\equiv$
\vspace{-1ex}
\begin{list}{}{} \item
\mbox{}\verb@html : m4_htmltarget@\\
\mbox{}\verb@@\\
\mbox{}\verb@@{\NWsep}
\end{list}
\vspace{-1.5ex}
\footnotesize
\begin{list}{}{\setlength{\itemsep}{-\parsep}\setlength{\itemindent}{-\leftmargin}}
\item \NWtxtMacroDefBy\ \NWlink{nuweb?}{?}\NWlink{nuweb?}{, ?}\NWlink{nuweb?}{, ?}\NWlink{nuweb?}{, ?}.
\item \NWtxtMacroRefIn\ \NWlink{nuweb?}{?}.

\item{}
\end{list}
\end{minipage}\vspace{4ex}
\end{flushleft}
The \HTML{} file depends on its source file and the graphics files.

Make lists of the graphics files and copy them.

\begin{flushleft} \small
\begin{minipage}{\linewidth}\label{scrap29}\raggedright\small
\NWtarget{nuweb?}{} $\langle\,${\itshape parameters in Makefile}\nobreak\ {\footnotesize {?}}$\,\rangle\equiv$
\vspace{-1ex}
\begin{list}{}{} \item
\mbox{}\verb@HTML_PS_FIG_NAMES=$(foreach fil,$(FIGFILES), m4_htmldocdir/$(fil).pstex)@\\
\mbox{}\verb@HTML_PST_NAMES=$(foreach fil,$(FIGFILES), m4_htmldocdir/$(fil).pstex_t)@\\
\mbox{}\verb@@{\NWsep}
\end{list}
\vspace{-1.5ex}
\footnotesize
\begin{list}{}{\setlength{\itemsep}{-\parsep}\setlength{\itemindent}{-\leftmargin}}
\item \NWtxtMacroDefBy\ \NWlink{nuweb?}{?}\NWlink{nuweb?}{, ?}\NWlink{nuweb?}{, ?}\NWlink{nuweb?}{, ?}\NWlink{nuweb?}{, ?}\NWlink{nuweb?}{, ?}\NWlink{nuweb?}{, ?}.
\item \NWtxtMacroRefIn\ \NWlink{nuweb?}{?}.
\item \NWtxtIdentsUsed\nobreak\  \verb@FIGFILES@\nobreak\ \NWlink{nuweb?}{?}.
\item{}
\end{list}
\end{minipage}\vspace{4ex}
\end{flushleft}
\begin{flushleft} \small
\begin{minipage}{\linewidth}\label{scrap30}\raggedright\small
\NWtarget{nuweb?}{} $\langle\,${\itshape impliciete make regels}\nobreak\ {\footnotesize {?}}$\,\rangle\equiv$
\vspace{-1ex}
\begin{list}{}{} \item
\mbox{}\verb@m4_htmldocdir/%.pstex : %.pstex@\\
\mbox{}\verb@        cp  $< $@{\tt @}\verb@@\\
\mbox{}\verb@@\\
\mbox{}\verb@m4_htmldocdir/%.pstex_t : %.pstex_t@\\
\mbox{}\verb@        cp  $< $@{\tt @}\verb@@\\
\mbox{}\verb@@\\
\mbox{}\verb@@{\NWsep}
\end{list}
\vspace{-1.5ex}
\footnotesize
\begin{list}{}{\setlength{\itemsep}{-\parsep}\setlength{\itemindent}{-\leftmargin}}
\item \NWtxtMacroDefBy\ \NWlink{nuweb?}{?}\NWlink{nuweb?}{, ?}\NWlink{nuweb?}{, ?}.
\item \NWtxtMacroRefIn\ \NWlink{nuweb?}{?}.

\item{}
\end{list}
\end{minipage}\vspace{4ex}
\end{flushleft}
Copy the nuweb file into the html directory.

\begin{flushleft} \small
\begin{minipage}{\linewidth}\label{scrap31}\raggedright\small
\NWtarget{nuweb?}{} $\langle\,${\itshape expliciete make regels}\nobreak\ {\footnotesize {?}}$\,\rangle\equiv$
\vspace{-1ex}
\begin{list}{}{} \item
\mbox{}\verb@m4_htmlsource : myscrapexamp.w@\\
\mbox{}\verb@        cp  myscrapexamp.w m4_htmlsource@\\
\mbox{}\verb@@\\
\mbox{}\verb@@{\NWsep}
\end{list}
\vspace{-1.5ex}
\footnotesize
\begin{list}{}{\setlength{\itemsep}{-\parsep}\setlength{\itemindent}{-\leftmargin}}
\item \NWtxtMacroDefBy\ \NWlink{nuweb?}{?}\NWlink{nuweb?}{, ?}\NWlink{nuweb?}{, ?}\NWlink{nuweb?}{, ?}\NWlink{nuweb?}{, ?}\NWlink{nuweb?}{, ?}\NWlink{nuweb?}{, ?}\NWlink{nuweb?}{, ?}.
\item \NWtxtMacroRefIn\ \NWlink{nuweb?}{?}.

\item{}
\end{list}
\end{minipage}\vspace{4ex}
\end{flushleft}
We also need a file with the same name as the documentstyle and suffix
\verb|.4ht|. Just copy the file \verb|report.4ht| from the tex4ht
distribution. Currently this seems to work.

\begin{flushleft} \small
\begin{minipage}{\linewidth}\label{scrap32}\raggedright\small
\NWtarget{nuweb?}{} $\langle\,${\itshape expliciete make regels}\nobreak\ {\footnotesize {?}}$\,\rangle\equiv$
\vspace{-1ex}
\begin{list}{}{} \item
\mbox{}\verb@m4_4htfildest : m4_4htfilsource@\\
\mbox{}\verb@        cp m4_4htfilsource m4_4htfildest@\\
\mbox{}\verb@@\\
\mbox{}\verb@@{\NWsep}
\end{list}
\vspace{-1.5ex}
\footnotesize
\begin{list}{}{\setlength{\itemsep}{-\parsep}\setlength{\itemindent}{-\leftmargin}}
\item \NWtxtMacroDefBy\ \NWlink{nuweb?}{?}\NWlink{nuweb?}{, ?}\NWlink{nuweb?}{, ?}\NWlink{nuweb?}{, ?}\NWlink{nuweb?}{, ?}\NWlink{nuweb?}{, ?}\NWlink{nuweb?}{, ?}\NWlink{nuweb?}{, ?}.
\item \NWtxtMacroRefIn\ \NWlink{nuweb?}{?}.

\item{}
\end{list}
\end{minipage}\vspace{4ex}
\end{flushleft}
Copy the bibliography.

\begin{flushleft} \small
\begin{minipage}{\linewidth}\label{scrap33}\raggedright\small
\NWtarget{nuweb?}{} $\langle\,${\itshape expliciete make regels}\nobreak\ {\footnotesize {?}}$\,\rangle\equiv$
\vspace{-1ex}
\begin{list}{}{} \item
\mbox{}\verb@m4_htmlbibfil : m4_anuwebdir/myscrapexamp.bib@\\
\mbox{}\verb@        cp m4_anuwebdir/myscrapexamp.bib m4_htmlbibfil@\\
\mbox{}\verb@@\\
\mbox{}\verb@@{\NWsep}
\end{list}
\vspace{-1.5ex}
\footnotesize
\begin{list}{}{\setlength{\itemsep}{-\parsep}\setlength{\itemindent}{-\leftmargin}}
\item \NWtxtMacroDefBy\ \NWlink{nuweb?}{?}\NWlink{nuweb?}{, ?}\NWlink{nuweb?}{, ?}\NWlink{nuweb?}{, ?}\NWlink{nuweb?}{, ?}\NWlink{nuweb?}{, ?}\NWlink{nuweb?}{, ?}\NWlink{nuweb?}{, ?}.
\item \NWtxtMacroRefIn\ \NWlink{nuweb?}{?}.

\item{}
\end{list}
\end{minipage}\vspace{4ex}
\end{flushleft}
Make a dvi file with \texttt{w2html} and then run
\texttt{htlatex}. 

\begin{flushleft} \small
\begin{minipage}{\linewidth}\label{scrap34}\raggedright\small
\NWtarget{nuweb?}{} $\langle\,${\itshape expliciete make regels}\nobreak\ {\footnotesize {?}}$\,\rangle\equiv$
\vspace{-1ex}
\begin{list}{}{} \item
\mbox{}\verb@m4_htmltarget : m4_htmlsource m4_4htfildest $(HTML_PS_FIG_NAMES) $(HTML_PST_NAMES) m4_htmlbibfil@\\
\mbox{}\verb@        cp w2html /home/paul/projecten/cltl/emoeco/myscrapexamp/bin@\\
\mbox{}\verb@        cd /home/paul/projecten/cltl/emoeco/myscrapexamp/bin && chmod 775 w2html@\\
\mbox{}\verb@        cd m4_htmldocdir && /home/paul/projecten/cltl/emoeco/myscrapexamp/bin/w2html myscrapexamp.w@\\
\mbox{}\verb@@\\
\mbox{}\verb@@{\NWsep}
\end{list}
\vspace{-1.5ex}
\footnotesize
\begin{list}{}{\setlength{\itemsep}{-\parsep}\setlength{\itemindent}{-\leftmargin}}
\item \NWtxtMacroDefBy\ \NWlink{nuweb?}{?}\NWlink{nuweb?}{, ?}\NWlink{nuweb?}{, ?}\NWlink{nuweb?}{, ?}\NWlink{nuweb?}{, ?}\NWlink{nuweb?}{, ?}\NWlink{nuweb?}{, ?}\NWlink{nuweb?}{, ?}.
\item \NWtxtMacroRefIn\ \NWlink{nuweb?}{?}.

\item{}
\end{list}
\end{minipage}\vspace{4ex}
\end{flushleft}
Create a script that performs the translation.



\begin{flushleft} \small
\begin{minipage}{\linewidth}\label{scrap35}\raggedright\small
\NWtarget{nuweb?}{} \verb@"w2html"@\nobreak\ {\footnotesize {?}}$\equiv$
\vspace{-1ex}
\begin{list}{}{} \item
\mbox{}\verb@#!/bin/bash@\\
\mbox{}\verb@# w2html -- make a html file from a nuweb file@\\
\mbox{}\verb@# usage: w2html [filename]@\\
\mbox{}\verb@#  [filename]: Name of the nuweb source file.@\\
\mbox{}\verb@`#' m4_header@\\
\mbox{}\verb@echo "translate " $1 >w2html.log@\\
\mbox{}\verb@NUWEB=/usr/local/bin/nuweb@\\
\mbox{}\verb@@\hbox{$\langle\,${\itshape filenames in w2html}\nobreak\ {\footnotesize \NWlink{nuweb?}{?}}$\,\rangle$}\verb@@\\
\mbox{}\verb@@\\
\mbox{}\verb@@\hbox{$\langle\,${\itshape perform the task of w2html}\nobreak\ {\footnotesize \NWlink{nuweb?}{?}}$\,\rangle$}\verb@@\\
\mbox{}\verb@@\\
\mbox{}\verb@@{\NWsep}
\end{list}
\vspace{-1.5ex}
\footnotesize
\begin{list}{}{\setlength{\itemsep}{-\parsep}\setlength{\itemindent}{-\leftmargin}}
\item \NWtxtIdentsUsed\nobreak\  \verb@nuweb@\nobreak\ \NWlink{nuweb?}{?}.
\item{}
\end{list}
\end{minipage}\vspace{4ex}
\end{flushleft}
The script is very much like the \verb|w2pdf| script, but at this
moment I have still difficulties to compile the source smoothly into
\textsc{html} and that is why I make a separate file and do not
recycle parts from the other file. However, the file works similar.


\begin{flushleft} \small
\begin{minipage}{\linewidth}\label{scrap36}\raggedright\small
\NWtarget{nuweb?}{} $\langle\,${\itshape perform the task of w2html}\nobreak\ {\footnotesize {?}}$\,\rangle\equiv$
\vspace{-1ex}
\begin{list}{}{} \item
\mbox{}\verb@@\hbox{$\langle\,${\itshape run the html processors until the aux file remains unchanged}\nobreak\ {\footnotesize \NWlink{nuweb?}{?}}$\,\rangle$}\verb@@\\
\mbox{}\verb@@\hbox{$\langle\,${\itshape remove the copy of the aux file}\nobreak\ {\footnotesize \NWlink{nuweb?}{?}}$\,\rangle$}\verb@@\\
\mbox{}\verb@@{\NWsep}
\end{list}
\vspace{-1.5ex}
\footnotesize
\begin{list}{}{\setlength{\itemsep}{-\parsep}\setlength{\itemindent}{-\leftmargin}}
\item \NWtxtMacroRefIn\ \NWlink{nuweb?}{?}.

\item{}
\end{list}
\end{minipage}\vspace{4ex}
\end{flushleft}
The user provides the name of the nuweb file as argument. Strip the
extension (e.g.\ \verb|.w|) from the filename and create the names of
the \LaTeX{} file (ends with \verb|.tex|), the auxiliary file (ends
with \verb|.aux|) and the copy of the auxiliary file (add \verb|old.|
as a prefix to the auxiliary filename).

\begin{flushleft} \small
\begin{minipage}{\linewidth}\label{scrap37}\raggedright\small
\NWtarget{nuweb?}{} $\langle\,${\itshape filenames in w2html}\nobreak\ {\footnotesize {?}}$\,\rangle\equiv$
\vspace{-1ex}
\begin{list}{}{} \item
\mbox{}\verb@nufil=$1@\\
\mbox{}\verb@trunk=${1%%.*}@\\
\mbox{}\verb@texfil=${trunk}.tex@\\
\mbox{}\verb@auxfil=${trunk}.aux@\\
\mbox{}\verb@oldaux=old.${trunk}.aux@\\
\mbox{}\verb@indexfil=${trunk}.idx@\\
\mbox{}\verb@oldindexfil=old.${trunk}.idx@\\
\mbox{}\verb@@{\NWsep}
\end{list}
\vspace{-1.5ex}
\footnotesize
\begin{list}{}{\setlength{\itemsep}{-\parsep}\setlength{\itemindent}{-\leftmargin}}
\item \NWtxtMacroRefIn\ \NWlink{nuweb?}{?}.
\item \NWtxtIdentsDefed\nobreak\  \verb@auxfil@\nobreak\ \NWlink{nuweb?}{?}\NWlink{nuweb?}{, ?}\NWlink{nuweb?}{, ?}, \verb@nufil@\nobreak\ \NWlink{nuweb?}{?}\NWlink{nuweb?}{, ?}\NWlink{nuweb?}{, ?}, \verb@oldaux@\nobreak\ \NWlink{nuweb?}{?}\NWlink{nuweb?}{, ?}\NWlink{nuweb?}{, ?}\NWlink{nuweb?}{, ?}, \verb@texfil@\nobreak\ \NWlink{nuweb?}{?}\NWlink{nuweb?}{, ?}\NWlink{nuweb?}{, ?}, \verb@trunk@\nobreak\ \NWlink{nuweb?}{?}\NWlink{nuweb?}{, ?}\NWlink{nuweb?}{, ?}\NWlink{nuweb?}{, ?}.\item \NWtxtIdentsUsed\nobreak\  \verb@indexfil@\nobreak\ \NWlink{nuweb?}{?}, \verb@oldindexfil@\nobreak\ \NWlink{nuweb?}{?}.
\item{}
\end{list}
\end{minipage}\vspace{4ex}
\end{flushleft}
\begin{flushleft} \small
\begin{minipage}{\linewidth}\label{scrap38}\raggedright\small
\NWtarget{nuweb?}{} $\langle\,${\itshape run the html processors until the aux file remains unchanged}\nobreak\ {\footnotesize {?}}$\,\rangle\equiv$
\vspace{-1ex}
\begin{list}{}{} \item
\mbox{}\verb@while@\\
\mbox{}\verb@  ! cmp -s $auxfil $oldaux @\\
\mbox{}\verb@do@\\
\mbox{}\verb@  if [ -e $auxfil ]@\\
\mbox{}\verb@  then@\\
\mbox{}\verb@   cp $auxfil $oldaux@\\
\mbox{}\verb@  fi@\\
\mbox{}\verb@  @\hbox{$\langle\,${\itshape run the html processors}\nobreak\ {\footnotesize \NWlink{nuweb?}{?}}$\,\rangle$}\verb@@\\
\mbox{}\verb@done@\\
\mbox{}\verb@@\hbox{$\langle\,${\itshape run tex4ht}\nobreak\ {\footnotesize \NWlink{nuweb?}{?}}$\,\rangle$}\verb@@\\
\mbox{}\verb@@\\
\mbox{}\verb@@{\NWsep}
\end{list}
\vspace{-1.5ex}
\footnotesize
\begin{list}{}{\setlength{\itemsep}{-\parsep}\setlength{\itemindent}{-\leftmargin}}
\item \NWtxtMacroRefIn\ \NWlink{nuweb?}{?}.
\item \NWtxtIdentsUsed\nobreak\  \verb@auxfil@\nobreak\ \NWlink{nuweb?}{?}\NWlink{nuweb?}{, ?}, \verb@oldaux@\nobreak\ \NWlink{nuweb?}{?}\NWlink{nuweb?}{, ?}.
\item{}
\end{list}
\end{minipage}\vspace{4ex}
\end{flushleft}
To work for \textsc{html}, nuweb \emph{must} be run with the \verb|-n|
option, because there are no page numbers.

\begin{flushleft} \small
\begin{minipage}{\linewidth}\label{scrap39}\raggedright\small
\NWtarget{nuweb?}{} $\langle\,${\itshape run the html processors}\nobreak\ {\footnotesize {?}}$\,\rangle\equiv$
\vspace{-1ex}
\begin{list}{}{} \item
\mbox{}\verb@$NUWEB -o -n $nufil@\\
\mbox{}\verb@latex $texfil@\\
\mbox{}\verb@makeindex $trunk@\\
\mbox{}\verb@bibtex $trunk@\\
\mbox{}\verb@htlatex $trunk@\\
\mbox{}\verb@@{\NWsep}
\end{list}
\vspace{-1.5ex}
\footnotesize
\begin{list}{}{\setlength{\itemsep}{-\parsep}\setlength{\itemindent}{-\leftmargin}}
\item \NWtxtMacroRefIn\ \NWlink{nuweb?}{?}.
\item \NWtxtIdentsUsed\nobreak\  \verb@bibtex@\nobreak\ \NWlink{nuweb?}{?}, \verb@makeindex@\nobreak\ \NWlink{nuweb?}{?}, \verb@nufil@\nobreak\ \NWlink{nuweb?}{?}\NWlink{nuweb?}{, ?}, \verb@texfil@\nobreak\ \NWlink{nuweb?}{?}\NWlink{nuweb?}{, ?}, \verb@trunk@\nobreak\ \NWlink{nuweb?}{?}\NWlink{nuweb?}{, ?}.
\item{}
\end{list}
\end{minipage}\vspace{4ex}
\end{flushleft}
When the compilation has been satisfied, run makeindex in a special
way, run bibtex again (I don't know why this is necessary) and then run htlatex another time.
\begin{flushleft} \small
\begin{minipage}{\linewidth}\label{scrap40}\raggedright\small
\NWtarget{nuweb?}{} $\langle\,${\itshape run tex4ht}\nobreak\ {\footnotesize {?}}$\,\rangle\equiv$
\vspace{-1ex}
\begin{list}{}{} \item
\mbox{}\verb@tex '\def\filename{{myscrapexamp}{idx}{4dx}{ind}} \input idxmake.4ht'@\\
\mbox{}\verb@makeindex -o $trunk.ind $trunk.4dx@\\
\mbox{}\verb@bibtex $trunk@\\
\mbox{}\verb@htlatex $trunk@\\
\mbox{}\verb@@{\NWsep}
\end{list}
\vspace{-1.5ex}
\footnotesize
\begin{list}{}{\setlength{\itemsep}{-\parsep}\setlength{\itemindent}{-\leftmargin}}
\item \NWtxtMacroRefIn\ \NWlink{nuweb?}{?}.
\item \NWtxtIdentsUsed\nobreak\  \verb@bibtex@\nobreak\ \NWlink{nuweb?}{?}, \verb@makeindex@\nobreak\ \NWlink{nuweb?}{?}, \verb@trunk@\nobreak\ \NWlink{nuweb?}{?}\NWlink{nuweb?}{, ?}.
\item{}
\end{list}
\end{minipage}\vspace{4ex}
\end{flushleft}
\paragraph{create the program sources}
\label{sec:createsources}

Run nuweb, but suppress the creation of the \LaTeX{} documentation.
Nuweb creates only sources that do not yet exist or that have been
modified. Therefore make does not have to check this. However,
``make'' has to create the directories for the sources if they
do not yet exist.
So, let's create the directories first.

\begin{flushleft} \small
\begin{minipage}{\linewidth}\label{scrap41}\raggedright\small
\NWtarget{nuweb?}{} $\langle\,${\itshape parameters in Makefile}\nobreak\ {\footnotesize {?}}$\,\rangle\equiv$
\vspace{-1ex}
\begin{list}{}{} \item
\mbox{}\verb@MKDIR = mkdir -p@\\
\mbox{}\verb@@\\
\mbox{}\verb@@{\NWsep}
\end{list}
\vspace{-1.5ex}
\footnotesize
\begin{list}{}{\setlength{\itemsep}{-\parsep}\setlength{\itemindent}{-\leftmargin}}
\item \NWtxtMacroDefBy\ \NWlink{nuweb?}{?}\NWlink{nuweb?}{, ?}\NWlink{nuweb?}{, ?}\NWlink{nuweb?}{, ?}\NWlink{nuweb?}{, ?}\NWlink{nuweb?}{, ?}\NWlink{nuweb?}{, ?}.
\item \NWtxtMacroRefIn\ \NWlink{nuweb?}{?}.
\item \NWtxtIdentsDefed\nobreak\  \verb@MKDIR@\nobreak\ \NWlink{nuweb?}{?}.
\item{}
\end{list}
\end{minipage}\vspace{4ex}
\end{flushleft}
\begin{flushleft} \small
\begin{minipage}{\linewidth}\label{scrap42}\raggedright\small
\NWtarget{nuweb?}{} $\langle\,${\itshape make targets}\nobreak\ {\footnotesize {?}}$\,\rangle\equiv$
\vspace{-1ex}
\begin{list}{}{} \item
\mbox{}\verb@DIRS = @\hbox{$\langle\,${\itshape directories to create}\nobreak\ {\footnotesize \NWlink{nuweb?}{?}}$\,\rangle$}\verb@@\\
\mbox{}\verb@@\\
\mbox{}\verb@$(DIRS) : @\\
\mbox{}\verb@        $(MKDIR) $@{\tt @}\verb@@\\
\mbox{}\verb@@\\
\mbox{}\verb@@{\NWsep}
\end{list}
\vspace{-1.5ex}
\footnotesize
\begin{list}{}{\setlength{\itemsep}{-\parsep}\setlength{\itemindent}{-\leftmargin}}
\item \NWtxtMacroDefBy\ \NWlink{nuweb?}{?}\NWlink{nuweb?}{, ?}\NWlink{nuweb?}{, ?}\NWlink{nuweb?}{, ?}.
\item \NWtxtMacroRefIn\ \NWlink{nuweb?}{?}.
\item \NWtxtIdentsDefed\nobreak\  \verb@DIRS@\nobreak\ \NWlink{nuweb?}{?}.\item \NWtxtIdentsUsed\nobreak\  \verb@MKDIR@\nobreak\ \NWlink{nuweb?}{?}.
\item{}
\end{list}
\end{minipage}\vspace{4ex}
\end{flushleft}
\begin{flushleft} \small
\begin{minipage}{\linewidth}\label{scrap43}\raggedright\small
\NWtarget{nuweb?}{} $\langle\,${\itshape make targets}\nobreak\ {\footnotesize {?}}$\,\rangle\equiv$
\vspace{-1ex}
\begin{list}{}{} \item
\mbox{}\verb@sources : myscrapexamp.w $(DIRS)@\\
\mbox{}\verb@        $(NUWEB) myscrapexamp.w@\\
\mbox{}\verb@@\\
\mbox{}\verb@jetty : sources@\\
\mbox{}\verb@        cd .. && mvn jetty:run@\\
\mbox{}\verb@@\\
\mbox{}\verb@@{\NWsep}
\end{list}
\vspace{-1.5ex}
\footnotesize
\begin{list}{}{\setlength{\itemsep}{-\parsep}\setlength{\itemindent}{-\leftmargin}}
\item \NWtxtMacroDefBy\ \NWlink{nuweb?}{?}\NWlink{nuweb?}{, ?}\NWlink{nuweb?}{, ?}\NWlink{nuweb?}{, ?}.
\item \NWtxtMacroRefIn\ \NWlink{nuweb?}{?}.
\item \NWtxtIdentsUsed\nobreak\  \verb@DIRS@\nobreak\ \NWlink{nuweb?}{?}.
\item{}
\end{list}
\end{minipage}\vspace{4ex}
\end{flushleft}
\section{References}
\label{sec:references}

\subsection{Literature}
\label{sec:literature}

\bibliographystyle{plain}
\bibliography{myscrapexamp}

\subsection{URL's}
\label{sec:urls}

\begin{description}
\item[Nuweb:] \url{nuweb.sourceforge.net}
\item[Apache Velocity:] \url{m4_velocityURL}
\item[Velocitytools:] \url{m4_velocitytoolsURL}
\item[Parameterparser tool:] \url{m4_parameterparserdocURL}
\item[Cookietool:] \url{m4_cookietooldocURL}
\item[VelocityView:] \url{m4_velocityviewURL}
\item[VelocityLayoutServlet:] \url{m4_velocitylayoutservletURL}
\item[Jetty:] \url{m4_jettycodehausURL}
\item[UserBase javadoc:] \url{m4_userbasejavadocURL}
\item[VU corpus Management development site:] \url{http://code.google.com/p/vucom} 
\end{description}

\section{Indexes}
\label{sec:indexes}


\subsection{Filenames}
\label{sec:filenames}


{\small\begin{list}{}{\setlength{\itemsep}{-\parsep}\setlength{\itemindent}{-\leftmargin}}
\item \verb@"../nuweb/bin/w2pdf"@ {\footnotesize {\NWtxtDefBy} \NWlink{nuweb?}{?}.}
\item \verb@"../scrape.py"@ {\footnotesize {\NWtxtDefBy} \NWlink{nuweb?}{?}.}
\item \verb@"Makefile"@ {\footnotesize {\NWtxtDefBy} \NWlink{nuweb?}{?}.}
\item \verb@"w2html"@ {\footnotesize {\NWtxtDefBy} \NWlink{nuweb?}{?}.}
\end{list}}

\subsection{Macro's}
\label{sec:macros}


{\small\begin{list}{}{\setlength{\itemsep}{-\parsep}\setlength{\itemindent}{-\leftmargin}}
\item $\langle\,$all targets\nobreak\ {\footnotesize \NWlink{nuweb?}{?}}$\,\rangle$ {\footnotesize {\NWtxtRefIn} \NWlink{nuweb?}{?}.}
\item $\langle\,$compile nuweb\nobreak\ {\footnotesize \NWlink{nuweb?}{?}}$\,\rangle$ {\footnotesize {\NWtxtRefIn} \NWlink{nuweb?}{?}.}
\item $\langle\,$default target\nobreak\ {\footnotesize \NWlink{nuweb?}{?}}$\,\rangle$ {\footnotesize {\NWtxtRefIn} \NWlink{nuweb?}{?}.}
\item $\langle\,$directories to create\nobreak\ {\footnotesize \NWlink{nuweb?}{?}}$\,\rangle$ {\footnotesize {\NWtxtRefIn} \NWlink{nuweb?}{?}.}
\item $\langle\,$expliciete make regels\nobreak\ {\footnotesize \NWlink{nuweb?}{?}\NWlink{nuweb?}{, ?}\NWlink{nuweb?}{, ?}\NWlink{nuweb?}{, ?}\NWlink{nuweb?}{, ?}\NWlink{nuweb?}{, ?}\NWlink{nuweb?}{, ?}\NWlink{nuweb?}{, ?}}$\,\rangle$ {\footnotesize {\NWtxtRefIn} \NWlink{nuweb?}{?}.}
\item $\langle\,$filenames in nuweb compile script\nobreak\ {\footnotesize \NWlink{nuweb?}{?}}$\,\rangle$ {\footnotesize {\NWtxtRefIn} \NWlink{nuweb?}{?}.}
\item $\langle\,$filenames in w2html\nobreak\ {\footnotesize \NWlink{nuweb?}{?}}$\,\rangle$ {\footnotesize {\NWtxtRefIn} \NWlink{nuweb?}{?}.}
\item $\langle\,$get program options\nobreak\ {\footnotesize \NWlink{nuweb?}{?}}$\,\rangle$ {\footnotesize {\NWtxtRefIn} \NWlink{nuweb?}{?}.}
\item $\langle\,$impliciete make regels\nobreak\ {\footnotesize \NWlink{nuweb?}{?}\NWlink{nuweb?}{, ?}\NWlink{nuweb?}{, ?}}$\,\rangle$ {\footnotesize {\NWtxtRefIn} \NWlink{nuweb?}{?}.}
\item $\langle\,$import modules in main program\nobreak\ {\footnotesize \NWlink{nuweb?}{?}}$\,\rangle$ {\footnotesize {\NWtxtRefIn} \NWlink{nuweb?}{?}.}
\item $\langle\,$make targets\nobreak\ {\footnotesize \NWlink{nuweb?}{?}\NWlink{nuweb?}{, ?}\NWlink{nuweb?}{, ?}\NWlink{nuweb?}{, ?}}$\,\rangle$ {\footnotesize {\NWtxtRefIn} \NWlink{nuweb?}{?}.}
\item $\langle\,$methods of the main program\nobreak\ {\footnotesize \NWlink{nuweb?}{?}}$\,\rangle$ {\footnotesize {\NWtxtRefIn} \NWlink{nuweb?}{?}.}
\item $\langle\,$parameters in Makefile\nobreak\ {\footnotesize \NWlink{nuweb?}{?}\NWlink{nuweb?}{, ?}\NWlink{nuweb?}{, ?}\NWlink{nuweb?}{, ?}\NWlink{nuweb?}{, ?}\NWlink{nuweb?}{, ?}\NWlink{nuweb?}{, ?}}$\,\rangle$ {\footnotesize {\NWtxtRefIn} \NWlink{nuweb?}{?}.}
\item $\langle\,$perform the task of w2html\nobreak\ {\footnotesize \NWlink{nuweb?}{?}}$\,\rangle$ {\footnotesize {\NWtxtRefIn} \NWlink{nuweb?}{?}.}
\item $\langle\,$print the testpost\nobreak\ {\footnotesize \NWlink{nuweb?}{?}}$\,\rangle$ {\footnotesize {\NWtxtRefIn} \NWlink{nuweb?}{?}.}
\item $\langle\,$remove the copy of the aux file\nobreak\ {\footnotesize \NWlink{nuweb?}{?}}$\,\rangle$ {\footnotesize {\NWtxtRefIn} \NWlink{nuweb?}{?}\NWlink{nuweb?}{, ?}.
}
\item $\langle\,$run tex4ht\nobreak\ {\footnotesize \NWlink{nuweb?}{?}}$\,\rangle$ {\footnotesize {\NWtxtRefIn} \NWlink{nuweb?}{?}.}
\item $\langle\,$run the html processors\nobreak\ {\footnotesize \NWlink{nuweb?}{?}}$\,\rangle$ {\footnotesize {\NWtxtRefIn} \NWlink{nuweb?}{?}.}
\item $\langle\,$run the html processors until the aux file remains unchanged\nobreak\ {\footnotesize \NWlink{nuweb?}{?}}$\,\rangle$ {\footnotesize {\NWtxtRefIn} \NWlink{nuweb?}{?}.}
\item $\langle\,$run the processors until the aux file remains unchanged\nobreak\ {\footnotesize \NWlink{nuweb?}{?}}$\,\rangle$ {\footnotesize {\NWtxtRefIn} \NWlink{nuweb?}{?}.}
\item $\langle\,$run the three processors\nobreak\ {\footnotesize \NWlink{nuweb?}{?}}$\,\rangle$ {\footnotesize {\NWtxtRefIn} \NWlink{nuweb?}{?}.}
\end{list}}

\subsection{Variables}
\label{sec:veriables}


{\small\begin{list}{}{\setlength{\itemsep}{-\parsep}\setlength{\itemindent}{-\leftmargin}}
\item \verb@all@: \underline{\NWlink{nuweb?}{?}}.
\item \verb@auxfil@: \underline{\NWlink{nuweb?}{?}}\NWlink{nuweb?}{, ?}, \underline{\NWlink{nuweb?}{?}}\NWlink{nuweb?}{, ?}.
\item \verb@bibtex@: \underline{\NWlink{nuweb?}{?}}\NWlink{nuweb?}{, ?}\NWlink{nuweb?}{, ?}.
\item \verb@boardnum@: \underline{\NWlink{nuweb?}{?}}\NWlink{nuweb?}{, ?}.
\item \verb@DIRS@: \underline{\NWlink{nuweb?}{?}}\NWlink{nuweb?}{, ?}.
\item \verb@fig2dev@: \underline{\NWlink{nuweb?}{?}}.
\item \verb@FIGFILENAMES@: \underline{\NWlink{nuweb?}{?}}.
\item \verb@FIGFILES@: \underline{\NWlink{nuweb?}{?}}\NWlink{nuweb?}{, ?}\NWlink{nuweb?}{, ?}.
\item \verb@indexfil@: \underline{\NWlink{nuweb?}{?}}\NWlink{nuweb?}{, ?}\NWlink{nuweb?}{, ?}.
\item \verb@makeindex@: \underline{\NWlink{nuweb?}{?}}\NWlink{nuweb?}{, ?}\NWlink{nuweb?}{, ?}.
\item \verb@MKDIR@: \underline{\NWlink{nuweb?}{?}}\NWlink{nuweb?}{, ?}.
\item \verb@nufil@: \underline{\NWlink{nuweb?}{?}}\NWlink{nuweb?}{, ?}, \underline{\NWlink{nuweb?}{?}}\NWlink{nuweb?}{, ?}.
\item \verb@nuweb@: \NWlink{nuweb?}{?}\NWlink{nuweb?}{, ?}\NWlink{nuweb?}{, ?}\NWlink{nuweb?}{, ?}, \underline{\NWlink{nuweb?}{?}}\NWlink{nuweb?}{, ?}.
\item \verb@oldaux@: \underline{\NWlink{nuweb?}{?}}\NWlink{nuweb?}{, ?}\NWlink{nuweb?}{, ?}, \underline{\NWlink{nuweb?}{?}}\NWlink{nuweb?}{, ?}.
\item \verb@oldindexfil@: \underline{\NWlink{nuweb?}{?}}\NWlink{nuweb?}{, ?}\NWlink{nuweb?}{, ?}.
\item \verb@pdf@: \NWlink{nuweb?}{?}\NWlink{nuweb?}{, ?}, \underline{\NWlink{nuweb?}{?}}\NWlink{nuweb?}{, ?}.
\item \verb@PDFT_NAMES@: \underline{\NWlink{nuweb?}{?}}\NWlink{nuweb?}{, ?}.
\item \verb@PDF_FIG_NAMES@: \underline{\NWlink{nuweb?}{?}}\NWlink{nuweb?}{, ?}.
\item \verb@PHONY@: \underline{\NWlink{nuweb?}{?}}\NWlink{nuweb?}{, ?}.
\item \verb@print@: \NWlink{nuweb?}{?}\NWlink{nuweb?}{, ?}, \underline{\NWlink{nuweb?}{?}}.
\item \verb@print_post@: \underline{\NWlink{nuweb?}{?}}\NWlink{nuweb?}{, ?}.
\item \verb@PST_NAMES@: \underline{\NWlink{nuweb?}{?}}.
\item \verb@PS_FIG_NAMES@: \underline{\NWlink{nuweb?}{?}}.
\item \verb@SUFFIXES@: \underline{\NWlink{nuweb?}{?}}.
\item \verb@texfil@: \underline{\NWlink{nuweb?}{?}}\NWlink{nuweb?}{, ?}, \underline{\NWlink{nuweb?}{?}}\NWlink{nuweb?}{, ?}.
\item \verb@trunk@: \underline{\NWlink{nuweb?}{?}}\NWlink{nuweb?}{, ?}, \underline{\NWlink{nuweb?}{?}}\NWlink{nuweb?}{, ?}\NWlink{nuweb?}{, ?}.
\item \verb@view@: \underline{\NWlink{nuweb?}{?}}.
\end{list}}

\end{document}

% Local IspellDict: british 

% LocalWords:  Webcom
